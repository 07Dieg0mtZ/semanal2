\documentclass[11pt]{beamer}

\usepackage{graphicx}
\usepackage[utf8]{inputenc}
%\usepackage[style=authoryear,backend=biber]{biblatex}



\titlegraphic{\includegraphics[height=4.5cm=]{tlotr.png}}


\title{Diego Martinez Ortiz}

\begin{document}

\begin{frame}
  \titlepage
\end{frame}

\begin{frame}{Teorema de Gödel: \cite{paginaweb}}
  El primer Teorema de Incompletitud de Gödel demostró que, no importa cuán completas y precisas sean las reglas, siempre habrá preguntas que no pueden resolverse dentro de un sistema, es decir que en cualquier sistema matemático suficientemente complejo (como el que usamos para la aritmética), hay afirmaciones que son verdaderas, pero que no podemos probar utilizando solo las reglas de ese sistema y ademas un sistema no puede garantizar su propia coherencia. Esto significa que las matemáticas, por muy lógicas y ordenadas que sean, no pueden probar por sí mismas que no contienen errores o contradicciones.
  Elegi este teorema por que me intriga la idea de que no importa que tanto sepamos, siempre habran cosas que esten mas alla de nuestra capacidad y comprension, representa, de cierto modo, el limite del conocimiento humano y que lo que creemos saber en realidad es incompleto.
\end{frame}


\begin{frame}
        Ecuación: $\frac{3}{4}x+9 =15$
        \begin{align*}
           \frac{3}{4}x+9 &=15 &\text{Despejamos}\\
           \frac{3}{4}x &= 15-9 &\text{Restamos}\\
           \frac{3}{4}x &= 6 &\text{Despejamos}\\
           x &= \frac{6}{\frac{3}{4}} &\text{Dividimos}\\
           x &= \frac{24}{3} &\text{Simplificamos}\\
           x &= 8 &\text{Resultado}
        \end{align*}
\end{frame}


\begin{frame}
  
  \frametitle{Bibliografia}

  \bibliographystyle{unsrt}
  
  \bibliography{referencias}
  
\end{frame}








\end{document}















